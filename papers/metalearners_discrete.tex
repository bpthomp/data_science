% !TEX TS-program = pdflatex
% !TEX encoding = UTF-8 Unicode

% This is a simple template for a LaTeX document using the "article" class.
% See "book", "report", "letter" for other types of document.

\documentclass[11pt]{article} % use larger type; default would be 10pt

\usepackage[utf8]{inputenc} % set input encoding (not needed with XeLaTeX)

%%% Examples of Article customizations
% These packages are optional, depending whether you want the features they provide.
% See the LaTeX Companion or other references for full information.

%%% PAGE DIMENSIONS
\usepackage{geometry} % to change the page dimensions
\geometry{a4paper} % or letterpaper (US) or a5paper or....
\geometry{margin=0.75in} % for example, change the margins to 2 inches all round
% \geometry{landscape} % set up the page for landscape
%   read geometry.pdf for detailed page layout information

\usepackage{graphicx} % support the \includegraphics command and options

% \usepackage[parfill]{parskip} % Activate to begin paragraphs with an empty line rather than an indent

%%% PACKAGES
\usepackage{booktabs} % for much better looking tables
\usepackage{array} % for better arrays (eg matrices) in maths
\usepackage{paralist} % very flexible & customisable lists (eg. enumerate/itemize, etc.)
\usepackage{verbatim} % adds environment for commenting out blocks of text & for better verbatim
\usepackage{subfig} % make it possible to include more than one captioned figure/table in a single float
% These packages are all incorporated in the memoir class to one degree or another...

%%% HEADERS & FOOTERS
\usepackage{fancyhdr} % This should be set AFTER setting up the page geometry
\pagestyle{fancy} % options: empty , plain , fancy
\renewcommand{\headrulewidth}{0pt} % customise the layout...
\lhead{}\chead{}\rhead{}
\lfoot{}\cfoot{\thepage}\rfoot{}

%%% SECTION TITLE APPEARANCE
\usepackage{sectsty}
\allsectionsfont{\sffamily\mdseries\upshape} % (See the fntguide.pdf for font help)
% (This matches ConTeXt defaults)

%%% ToC (table of contents) APPEARANCE
\usepackage[nottoc,notlof,notlot]{tocbibind} % Put the bibliography in the ToC
\usepackage[titles,subfigure]{tocloft} % Alter the style of the Table of Contents
\renewcommand{\cftsecfont}{\rmfamily\mdseries\upshape}
\renewcommand{\cftsecpagefont}{\rmfamily\mdseries\upshape} % No bold!

%%% END Article customizations

%%% The "real" document content comes below...

\title{Discretized Targeting Using Metalearners for Heterogeneous Treatment Effects}
\author{Ben Thompson $^1$}
\date{$^1$ Arena Technologies \\
	\today }
	 
\begin{document}
\maketitle
\begin{abstract}
Metalearners for estimating heterogeneous treatment effects are used ubiquitously to estimate 
Conditional Average Treatment Effects (CATEs) in a variety of applications,and increasingly, for targeting 
based on causal estimates; however, when taking discrete actions, such as whether or not to give a 
costly dichotomous treatment to a user based on a predicted effect informed by these metalearners, 
there is considerable ambiguity aobut what to do due to inherently aggregated backtesting. 
This paper proposes a solution to this discretization problem that balances the bias-variance 
tradeoff that arises from proposing cutoffs based on coarser vs finer grained bins of backtesting
results.
\end{abstract}

\section{Introduction}

Metalearners and other models for heterogeneous treatment effecst are widely used for a variety of 
applications that are increasingly not limited to the ex-post analysis of experimental data. 
In medical and industry use cases, \textit{predicted} treatment effects, before any 
intervention is made, are either used directly or as features in models 
(Pan, et al, https://www.ncbi.nlm.nih.gov/pmc/articles/PMC9291969/). However, while the model outputs are 

\subsection{Existing Literature}

More text.

\section{Problem Description}

\section{Solution}

\section{Simulated Results}

\end{document}
